hello Lorena, thanks for this. I must admit that "computationally
enabled teaching and learning" as it stands does exclude assessment. I
would say that the production of hard-copy tables for use in written
examinations [rather than photocopying tables from the back of a book,
as many people do] should count as computationally-created educational
resources.

But quite apart from that, I would argue that hardcopy log tables do
have considerable intrinsic merit as a day-to-day educational
resource:

* Extracting information from tables is a non-trivial yet useful (and
teachable!) skill

* A hardcopy table shows the overall structure of functions such as
log and sin in a very different way from a graph: such insight is only
obtainable from tabular forms, and hardcopy emphasises such
interpretations.

* The top few rows of a log table, for
example, are a striking illustration of the problematic behaviour of log(x)
for small values of x

* Log and trig tables have a rich and interesting
history, surely a worthy "interest" component of a general mathematics
education

* The very existence of a log table explicitly highlights to
students that the logarithm function is difficult to evaluate:
sufficiently difficult, in fact, to make it worthwhile for someone,
somewhere, to produce a table. Perhaps some students will value their
electronic calculators more highly as a result

* By using hardcopy tables, students can gain a deeper understanding of
the concepts underlying logarithms. This is because they are required
to think through the relationship between the logarithmic values and
the numbers they represent, rather than simply inputting values into a
calculator

* Hardcopy tables promote approximation skills by only having four
significant figures available: incidentally, this conveys the
ultra-important yet often-ignored fact that "only" four sig figs is
plenty accurate enough for many purposes; observing exactly how
working to only four sig figs lets one down is surely of high
educational value

* Restricting to four significant figures also helps students
understand the limitations of measurement.  In many fields, it is not
possible to measure values with absolute precision.

* Using a table of logarithms with four significant figures can help
students understand that measurements and calculations always involve
some degree of approximation, and can help them develop a better
understanding of the limitations of measurement

* Using four significant figures, as per the tables, corresponds to
5cm (sic) error in 1km; surely a startling and memorable metaphor

* The "proportional parts" section of a typical table gives a very
direct lesson about the difficulties and motivations for interpolation
as a general mathematical operation [incidentally, about 99% of the
conceptual difficulty of producing the tables is in the optimization
of the PPs]

* Interpreting results accurately: after performing calculations,
students must pay close attention to the units and significant figures
of the results to ensure that they are accurate and meaningful. This
requires attention to detail and a level of understanding that,
arguably, electronic calculators suppress

* Occasionally, when I cannot sleep, I try to verify an entry in a log
table using only paper and pencil.  This is a very sobering exercise,
one that can only enhance respect for the diligent and dedicated
workers, often overlooked, who originally made the tables.  The first
"computers" were of course real people, and this is an important part
of the history of mathematics


There does not seem to be any open-source version of log tables
available, surely an omission of interest to JOSE. From a research
perspective, it turns out that there are non-trivial mathematical
details in the creation of such tables and these are set out in
exhaustive detail in the repo.

Have I made a more convincing case?


